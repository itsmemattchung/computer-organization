\documentclass{article}
\usepackage[utf8]{inputenc}
\title{Lesson 6 - Computer Organization - CS1410}
\author{Matt Chung}
\date{September 16, 2017}
\usepackage{tikz}
\usepackage{verbatim}
\renewcommand{\thesubsection}{\thesection.\alph{subsection}}

\begin{document}
\maketitle

\section{}
The primary function of a CPU is to execute program instructions.  It does this by repeatedly following a pattern: fetch program instructions and decode these instructions and carry out these instructions on the correct data.  The CPU is able to carry out these instructions with the help of the control unit, a module responsible for sequencing operations and ensuring the right data is in the right place, at the right time.  The data travels along the datapath, a network of storage units and arithmetic units and logic units, all connected through a bus.

\section{}
As explained in the previous question, the CPU typically follows a pattern (i.e. fetch, decode, operate). However, at the end of each cycle, the CPU checks if an interrupt is present, waiting to be processed.  If no interrupt is present, the CPU continues normally, fetching and executing and operating. But if an interrupt has occurred, the CPU must decide whether or not it must process this interrupt, which depends on the type of interrupt: nonmaskable or maskable. If nonmaskable, the CPU must process the interrupt immediately; if not, the interrupt is considered non-critical and the CPU can elect whether or not to suspend the interrupt or process it.  Either way, after the interrupt has been processed, the CPU continues following it's normal procedure

Let's assumine for a moment that the CPU just detected an interrupt, what does the CPU do next? First, it needs to save the state of the current program, storing all it's values (in either other registers or memory). Then, it processes the interrupt; how it processes the interrupt depends on interrupt service routine, the block of code associated with the type of interrupt.

Once the interrupt has been processed, it loads the previous program, setting the PC with the next instruction.

\setcounter{section}{3}
\section{}

How many bits are required to address a 4M x 16 main memory if a) main memory is byte addressable and b) main memory is word addressable ?

First, let's define a \textbf{word}.  A word is the number of bits in the width of a memory. For example, in a 4M x 16 memory, the word size is 16 bits. In a 4M x 8, the word size is 8 bits.

\subsection{4M x 16 - byte addressable}

\textbf{Solution:} $2^{23}$

To determine the number of bits we need to address the memory, we breakdown the configuration into a formula: $2^{2} \times 2^{20} \times 2^1$. The first number represent the $4$ in $4M$; $2^{20}$ for million, the $M$ in $4M$; finally, the last digit represents the number of bytes in a word; since there are 8 bits in a byte and each row has 16 bits, we get $2^1$. Therefore, in total, we get $2^{23}$.

\subsection{4M x 16 - word addressable}

\textbf{Solution:} $2^{22}$

Similar to the byte addressable, we formulate the number of bits: $2^2 \times 2^22$. As you can see, unlike the previous question, we omit the third number, since our memory address is word addressable, the entire width (16 bits) of the memory

\setcounter{section}{9}

\section{}

2048 bytes containing 64 x 8 RAM chips and assuming byte-addressable memory, which of the seven diagrams suits best?

\textbf{Solution: } a)  2 bits for chip select and 8 bits for address on chip.

Before we begin this problem, and assuming byte-addressable memory, we can determine the number of bits needed in total.  Let's convert 2048 from base 10 to base 2; 2048 in base 2 gives us $2^11$.  This number's exponent is equivalent to the total number of bits we need for addressing.  So, the number of bits for the chips and the number of bits for the addresses on each chip must equate to 11 bits.

And the best way (in my opinion) to answer this question is to break it down into two, discrete questions:

\begin{itemize}
    \item How many bits are needed to select the right chip?
    \item How many bits are needed for all the addresses for a given chip
\end{itemize}

There are 8 bits in a byte and since each RAM chip contains 64 rows of 8 bits, we can infer that the chip consists of 64 bytes.  We can take this number, and divide it by 2048, our total number of bytes. This division calculates the number of chips: 32. 
To select the right chip, we need 5 bits (i.e. $2^{5}$. 

Now, let's move on to the number of bits needed for the addresses on each chip. Well, since each chip contains 64 rows, we need 6 bits (i.e. $2^6$). 

In summary, we need 2 bits to identify the correct chip and another 8 bits to identify the right address within that chip.

\setcounter{section}{0}

\section{Additional Question}

800 MHz clock, how many in nanoseconds?

\textbf{Solution:} 1.25 ns

To convert the clock frequency (i.e. 800 mhz) into clock cycle time, we simply take the recipricol. So, 1/800,000,000 equals 1.25 nanoseconds.




\end{document}
